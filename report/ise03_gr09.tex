\documentclass[a4paper]{article}

%% Language and font encodings
\usepackage[english]{babel}
\usepackage[utf8x]{inputenc}
\usepackage[T1]{fontenc}

%% Sets page size and margins
\usepackage[a4paper,top=3cm,bottom=2cm,left=3cm,right=3cm,marginparwidth=1.75cm]{geometry}

%% Useful packages
\usepackage{amsmath}
\usepackage{amsfonts} %mathematical fields fonts
\usepackage{graphicx}
\usepackage{steinmetz} %for complex numbers notation
\usepackage{float} %images held
\usepackage[colorinlistoftodos]{todonotes}
\usepackage[colorlinks=true, allcolors=blue]{hyperref}
\usepackage{xcolor} %code ambient colours
%\usepackage[colorlinks]{hyperref} %package to change colors
\usepackage{listings} %code snippet inside text
\lstset{basicstyle=\ttfamily,
	showstringspaces=false,
	commentstyle=\color{red},
	keywordstyle=\color{blue}
}


%% Defining colours as emacs in code snippet
\definecolor{Orange}{rgb}{255,128,0}
\definecolor{lBlue}{rgb}{51,255,255}
\definecolor{lGreen}{rgb}{51,255,153}

\title{Integrazione di Sistemi Embedded\\ Laboratorio 03}
\author{Matteo Perotti 251453\\ Giuseppe Puletto\\ Luca Romani 255244\\ Giuseppe Sarda}

\begin{document}

\maketitle

\newpage

\section{Esercizio 1}
Il programma si compone di tre files .c e tre header files.
Un ciclo infinito si occupa di leggere il prossimo carattere in arrivo salvandolo nella posizione più a destra
di un array di caratteri lungo quanto il numero di caratteri del comando previsto più lungo. Prima della
memorizzazione del carattere i dati in ogni cella dell'array vengono spostati di una cella a sinistra.

Ogni volta che un nuovo carattere è stato memorizzato, la funzione readCommand analizza il buffer per vedere
se può essere arrivato un comando valido. Questa operazione è svolta andando a controllare la cella dell'array
nella quale, se fosse stato ricevuto il comando più corto (draw a point), sarebbe salvato il carattere
identificativo del comando stesso. In caso di mancato riscontro viene analizzata la prima cella dell'array, ovvero
quella in cui potrebbe essere presente una delle altre due lettere identificative di un comando valido.

In caso di riscontro, i parametri relativi al comando vengono salvati nella struttura apposita, convertendo ogni 
carattere in intero. In seguito viene fatto un controllo sui dati salvati, per vedere se sono parametri validi,
ossia se ogni coordinata è un numero compreso tra 0 e 127, e se il "modo" è un numero compreso tra 0 e 2.

Se il comando non viene riconosciuto o se i parametri non sono corretti, la funzione restituisce 0 e viene letto il
prossimo carattere.

Nel caso in cui il comando venga riconosciuto e possegga parametri validi, allora viene chiamata la funzione 
corrispondente.
	\subsection{main.c}
		Il file main.c contiene la definizione della funzione main, la funzione principale del programma. 
		Essa si occupa di definire l'array di char cmdBuffer in cui viene memorizzato un carattere alla volta e
		la struttura in cui i parametri del comando vengono memorizzati.
		Prima del ciclo, l'array di caratteri viene inizializzato con null characters per evitare che valori
		casuali possano portare a riconoscimenti errati di comandi mai ricevuti.

		Nel ciclo infinito viene traslato il buffer di arrivo e viene salvato il nuovo carattere nell'ultima
		posizione.
		Il buffer è quindi passato alla funzione readCommand insieme alla struttura; a seconda dell'intero
		ritornato da quest'ultima funzione viene controllato il comando valido ed invocata la funzione 
		corrispondente, oppure il ciclo riprende da capo.

		Nel caso in cui la funzione readCommand dovesse ritornare una lettura corretta ma non fosse stato
		salvato un comando valido nella struttura, la funzione main restituisce un 1 per segnalare un errore.
	\subsection{read.c and read.h}
		Nel file header read.h sono presenti i comandi $\#$define per aumentare la leggibilità e la manutenibilità
		del codice, insieme con la dichiarazione del tipo "basicCmd", ossia un tipo-struttura  utile per la variabile
		in cui saranno salvati i vari parametri del comando, e la dichiarazione delle funzioni relative alla
		lettura del prossimo carattere dal periferico di input, al riconoscimento del comando e al salvataggio
		dei relativi parametri e alla conversione di un carattere ad intero. 
		\paragraph{readChar(void)}
			La funzione ritorna il carattere che viene letto dal periferico di input, in questo caso lo standard input.
			Viene effettuata una chiamata a funzione "getc()" con arcomento "stdin" (il tutto corrisponde a "getchar()").
		\paragraph{char2int(char charIn)}
			La funzione riceve un carattere in ingresso e restituisce uno short int che corrisponde alla cifra 
			rappresentata in ASCII. Se il carattere non è una cifra da 0 a 9, viene restituito 10, perché è un
			valore che aiuta il parser a capire se i parametri dell'ipotetico comando sono validi oppure no.
		\paragraph{readCommand(char* cmdBuffer, basicCmd* cmdStruc$\_$pt)}
			La funzione riceve l'array buffer in cui è salvato il possibile comando, insieme con la struttura dati in
			cui i parametri del comando vengono salvati.

			La funzione si occupa di controllare nelle posizioni chiave se è presente un identificativo del comando.
			Se è presente, la struttura viene aggiornata grazie all'utilizzo di char2int(char charIn).
			In seguito viene eseguito un controllo sui parametri appena aggiornati: se sono validi, allora viene
			ritornato un 1.

			In caso il comando non sia riconosciuto come completamente valido, viene ritornato un 1.
      \subsection{draw.c, draw.h and shared.h}
      Il file draw.c contiene la definizione e la descrizione dell'interfaccia delle funzioni drawPoint, drawLine e drawEllipse e la direttiva \#include al file locale draw.h.
      IL file draw.h contiene la dichiarazione delle funzioni drawPoint, drawLine e drawEllipse e la definizione di tre macro, tre variabili di nome DRAW\_MODE\_CLEAR, DRAW\_MODE\_SET e DRAW\_MODE\_XOR.
      draw.h ha anche una direttiva \#include al file locale shared.h e una al file di sistema stdio.h.
      IL file shared.h contiene la definizione delle macro rowsFrame, colsFrame e wordPixels e la definizione della variabile globale frameBuffer,
      corrispondende allo schermo virtuale su cui disegnare punti, ellissi e linee a richiesta. frameBuffer viene poi definito nel main.c
		\paragraph{int drawPoint(int x, int y, int m)} La funzione drawPoint è quella più basilare delle tre disponibili per disegnare. Come dichiarato dal prototipo nel file header draw.h, la funzione riceve tre interi, di cui i primi due sono le coordinate x e y del frame buffer. Dato che, dal punto di vista del piano cartesiano le y rappresentano l'altezza in cui si troverà il punto e le x la distanza orizzontale dall'origine, nel momento in cui si passano le due coordinate alla funzione è necessario che queste vengano invertite per puntare alla cella corretta del frameBuffer, che è una matrice di caratteri da 128 righe e 16 colonne da 1 Byte ciascuna. Per cui la variabile y corrisponderà alla i-esima riga della matrice, mentre la variabile x dovrà essere divisa per 8 per capire a quale degli 8 bit del j-esimo carattere si vuole puntare. Bisogna tener presente che le coordinate passate a questa funzione sono state controllate da readCommand(), di conseguenza si assume che i valori rispettivamente di x e y sono compresi tra 0-15 e 0-127.\\
		Facendo sempre riferimento ad un ipotetico piano cartesiano, se si vuole scrivere l'i-esimo pixel della matrice a partire dal basso occore complementare la coordinata y, sottraendola al numero 127. Siccome nel linguaggio C non esiste la possibilità di accedere direttamente ad uno specifico bit di un carattere è necessario effettuare operazioni ``bitwise" combinando le funzioni booleane e lo shifting di bit, che in questo caso avviene da destra verso sinistra di una quantità pari al resto della divisione di x per otto (il MSB è l'ultimo a sinistra).
		\newline
		Se l'intero m che viene passato alla funzione, corrisponde ad una delle tre modalità di scrittura del buffer definite nell'header ``shared.h" allora la funzione modificherà l'opportuno bit nel modo desiderato e ritornerà zero, altrimenti verrà ritornato un uno per segnalare la presenta di un errore. 
		\paragraph{int drawLine(unsigned int x1,unsigned int y1,unsigned int x2,unsigned int y2,int m)}
                La funzione drawLine descrive, dati due estremi, un segmento all'interno del frameBuffer.
                \newline
                Riceve, come da prototipo, le coordinate dei due punti (x1,y1) e (x2,y2), ed infine la modalità di disegno.
                Ritorna il valore intero '0' se la scrittura all'interno del frameBuffer o '1' se le coordinate degli estremi sono superiori all dimensioni massime dei ``pixel'' scrivibili.
                L'algoritmo utilizzato per la scrittura del codice è stato ricavato da un documento reperibile al seguente indirizzo web a pagina 6:\newline
                \href{http://www.idav.ucdavis.edu/education/GraphicsNotes/Bresenhams-Algorithm.pdf}\newline
                Lo pseudo-codice si basa sull'algoritmo di Bresenham implementato con numeri interi.
                Tuttavia nel pdf viene considerato il punto P$_{1}$(x$_{1}$,y$_{1}$) sempre l'estremo di partenza, dando per scontato che le ordinate e coordinate siano di modulo inferiore a quelle del punto P$_{2}$(x$_{2}$,y$_{2}$).
                È stata dunque necessaria una rivisitazione per considerare tutti gli altri casi e permettere un corretto funzionamento del codice.
                La funzione richiede, infine, l'utilizzo di 6 \textit{int} e 1 \textit{char}.
		\paragraph{int drawEllipse(int xc, int xy, int dx, int dy, int m)}
		La funzione drawEllipse riceve cinque variabili intere come parametri:
                le cordinate cartesiane del centro dell'ellisse, i diametri orizontale e verticale e il modo attraverso cui scrivere su un array di char.
		
		drawEllipse è definita nel file draw.c, dichiarata nel file draw.h.
                Fa uso di tre macro xmax, ymax e wordPixels, definite le prime due in read.h, l'ultima in shared.h, tre variabili intere di cui la funzione fa ampio uso.
		
		Essa si serve dell'algoritmo di Bresenham per disegnare sull'array di char.
                Il codice dell'algoritmo è stato ricavato correggendo e modificando la porzione di righe di codice trovata alla pagina web
                \href{https://sites.google.com/site/ruslancray/lab/projects/bresenhamscircleellipsedrawingalgorithm/bresenham-s-circle-ellipse-drawing-algorithm}{Bresenham's circle or ellipse drawing algorithm}.
                L'algoritmo trovato viene corretto usando in totale due variabili locali di tipo int in meno, non usando la funzione DrawPixel()
                ed ottimizzando l'algoritmo in modo che non scriva un pixel più di una volta. Quest'ultima correzione è la più importante:
                senza questa, drawEllipse non sarebbe stata in grado di disegnare secondo il draw mode xor.
                Altra importante correzione riguarda il controllo della posizione del pixel da scrivere.
                Se questa cade al di fuori dello schermo virtuale a disposizione, i bit dell'array di char, il pixel non viene disegnato.
		
		La funzione ritorna uno zero.

\section{Esercizio 2}
	\subsection{Script}
	Per automatizzare il processo di compilazione e generazione del file eseguibile è stato elaborato lo script ``universalDrawerScript.sh", che quando viene lanciato riceve come argomento il tipo di compilatore che si vuole usare: gcc per piattaforme x86-64, arm-none-eabi-gcc per piattaforme ARM prive di sistema operativo. Nel caso in cui il primo argomento non sia tra quelli attesi o sia assente, lo script termina immediatamente l'esecuzione.
	\newline
	Una volta scelta la piattaforma target lo script compila i file draw.c, read.c e main.c (con opzione ``-Wall" attiva per mostrare tutti i warning), viene eseguito il linking con il compilatore scelto e tutti i file oggetto vengono spostati in una cartella generata dallo script che può avere due nomi differenti:
	\begin{itemize}
		\item x86\textunderscore64\textunderscore platform 
		\item arm\textunderscore platform 
	\end{itemize}
in entrambi i casi all'interno della cartella corrispondente si troveranno i file oggetto e un file log (universalDrawerArmLog per ARM e universalx86\textunderscore 64Log per x86\textunderscore 64) al cui interno sono contenuti i risultati dei comandi ``nm" e ``size" rispettivamente.
\newline
Dato che per la piattaforma ARM non c'è un sistema operativo che si interfaccia con il programma scritto  (direttiva data con ``arm-none-eabi-gcc"), è necessario aggiungere l'opzione ``specs=nosys.specs" al compilatore per dirgli di non preoccuparsi delle funzioni di sistema dato che è compito di chi programmerà la piattaforma gestire i vari ritorni e funzionalità del programma.

Di seguito viene riportato il contenuto dei file di log per ciascuna piattaforma:
		\paragraph{x86$\_$64}
		\begin{center}
		\begin{lstlisting}
		text	   data	    bss	    dec	    hex	filename
		400	      0	      0	    400	    190	main.o
		1825	      0	      0	   1825	    721	read.o
		6017	      0	      0	   6017	   1781	draw.o
		# size output
		
		
		main.o:
		U drawEllipse
		U drawLine
		U drawPoint
		U _GLOBAL_OFFSET_TABLE_
		0000000000000000 T main
		U readChar
		U readCommand
		U __stack_chk_fail
		
		draw.o:
		00000000000016d0 T abs
		0000000000000437 T drawEllipse
		000000000000019a T drawLine
		0000000000000000 T drawPoint
		0000000000000800 C frameBuffer
		
		read.o:
		0000000000000015 T char2int
		U _GLOBAL_OFFSET_TABLE_
		U _IO_getc
		0000000000000000 T readChar
		0000000000000044 T readCommand
		U stdin
		#nm output
		\end{lstlisting}	
		\end{center}
		
		\paragraph{ARM}
		\begin{center}
		\begin{lstlisting}
		text	   data	    bss	    dec	    hex	filename
		396	      0	      0	    396	    18c	main.o
		2656	      0	      0	   2656	    a60	read.o
		8096	      0	      0	   8096	   1fa0	draw.o
		# size output
			
			
		main.o:
		00000000 t $a
		U drawEllipse
		U drawLine
		U drawPoint
		00000000 T main
		U readChar
		U readCommand
			
		draw.o:
		00000000 t $a
		00000220 t $a
		000005e0 t $a
		000016b4 t $a
		00001f64 T abs
		0000021c t $d
		000005dc t $d
		000016b0 t $d
		000005e0 T drawEllipse
		00000220 T drawLine
		00000000 T drawPoint
		00000800 C frameBuffer
		
		read.o:
		00000000 t $a
		00000088 t $a
		00000088 T char2int
		00000084 t $d
		U _impure_ptr
		00000000 T readChar
		000000e0 T readCommand
		U __srget_r
		#nm output
		\end{lstlisting}
		\end{center}
		
		\paragraph{Script}
		\begin{center}
		\begin{lstlisting}[language=bash, basicstyle=\small]
		#!/bin/bash
		
		#  compiling commands
		#  This script generates a directory and a log file
		#  according to the first argument provided:
		#  if it is gcc, the script uses gcc compiler for x86_64 platforms
		#  and creates object files in the x86_64_platform directory. The log
		#  file cointains informations regarding memory allocation of the code
		#  Same behaviour if first argument is arm-none-eabi-gcc
		#  Exit 1 is asserted in the wrong or missing argument case
		
		
		if [[ $1 = gcc ]]
		then
		
		$1 -Wall -c read.c 
		$1 -Wall -c draw.c 
		$1 -Wall -c main.c
		$1 -o universalDrawer main.o read.o draw.o
		chmod +x universalDrawer
		
		mkdir x86_64_platform
		mv *.o x86_64_platform
		cd x86_64_platform
		
		size main.o read.o draw.o | cat > sizeFile 
		printf "# size output\n\n" >> sizeFile
		nm main.o draw.o read.o | cat > nmFile
		cat sizeFile nmFile > universalDrawerx86_64Log
		printf "#nm output" >> universalDrawerx86_64Log
		
		rm sizeFile nmFile
		cd ..
		
		elif [[ $1 = arm-none-eabi-gcc ]]
		then
		$1 -Wall --specs=nosys.specs -c read.c 
		$1 -Wall --specs=nosys.specs -c draw.c 
		$1 -Wall --specs=nosys.specs -c main.c
		$1 --specs=nosys.specs -o universalDrawer main.o read.o draw.o
		chmod +x universalDrawer
		
		mkdir arm_platform
		mv *.o arm_platform
		cd arm_platform
		
		size main.o read.o draw.o | cat > sizeFile 
		printf "# size output\n\n" >> sizeFile
		nm main.o draw.o read.o | cat > nmFile
		cat sizeFile nmFile > universalDrawerArmLog
		printf "#nm output" >> universalDrawerArmLog
		
		rm sizeFile nmFile
		cd ..
		
		else
		
		echo Invalid argument 1, you must inser gcc or arm-none-eabi-gcc
		exit 1
		
		fi
		\end{lstlisting}
		\end{center}
\end{document}
