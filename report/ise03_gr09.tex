\documentclass[a4paper]{article}

%% Language and font encodings
\usepackage[english]{babel}
\usepackage[utf8x]{inputenc}
\usepackage[T1]{fontenc}

%% Sets page size and margins
\usepackage[a4paper,top=3cm,bottom=2cm,left=3cm,right=3cm,marginparwidth=1.75cm]{geometry}

%% Useful packages
\usepackage{amsmath}
\usepackage{amsfonts} %mathematical fields fonts
\usepackage{graphicx}
\usepackage{steinmetz} %for complex numbers notation
\usepackage{float} %images held
\usepackage[colorinlistoftodos]{todonotes}
\usepackage[colorlinks=true, allcolors=blue]{hyperref}
\usepackage{xcolor} %code ambient colours
\usepackage[colorlinks]{hyperref} %package to change colors
\usepackage{listings} %code snippet inside text

%% Defining colours as emacs in code snippet
\definecolor{Orange}{rgb}{255,128,0}
\definecolor{lBlue}{rgb}{51,255,255}
\definecolor{lGreen}{rgb}{51,255,153}

\title{Integrazione di Sistemi Embedded\\ Laboratorio 03}
\author{Matteo Perotti 251453\\ Giuseppe Puletto\\ Luca Romani\\ Giuseppe Sarda}

\begin{document}

\maketitle

\newpage

\section{Esercizio 1}
Il programma si compone di tre files .c e tre header files.
Un ciclo infinito si occupa di leggere il prossimo carattere in arrivo salvandolo nella posizione più a destra
di un array di caratteri lungo quanto il numero di caratteri del comando previsto più lungo. Prima della
memorizzazione del carattere i dati in ogni cella dell'array vengono spostati di una cella a sinistra.

Ogni volta che un nuovo carattere è stato memorizzato, la funzione readCommand analizza il buffer per vedere
se può essere arrivato un comando valido. Questa operazione è svolta andando a controllare la cella dell'array
nella quale, se fosse stato ricevuto il comando più corto (draw a point), sarebbe salvato il carattere
identificativo del comando stesso. In caso di mancato riscontro viene analizzata la prima cella dell'array, ovvero
quella in cui potrebbe essere presente una delle altre due lettere identificative di un comando valido.

In caso di riscontro, i parametri relativi al comando vengono salvati nella struttura apposita, convertendo ogni 
carattere in intero. In seguito viene fatto un controllo sui dati salvati, per vedere se sono parametri validi,
ossia se ogni coordinata è un numero compreso tra 0 e 127, e se il "modo" è un numero compreso tra 0 e 2.

Se il comando non viene riconosciuto o se i parametri non sono corretti, la funzione restituisce 0 e viene letto il
prossimo carattere.

Nel caso in cui il comando venga riconosciuto e possegga parametri validi, allora viene chiamata la funzione 
corrispondente.
	\subsection{main.c}
		Il file main.c contiene la definizione della funzione main, la funzione principale del programma. 
		Essa si occupa di definire l'array di char cmdBuffer in cui viene memorizzato un carattere alla volta e
		la struttura in cui i parametri del comando vengono memorizzati.
		Prima del ciclo, l'array di caratteri viene inizializzato con null characters per evitare che valori
		casuali possano portare a riconoscimenti errati di comandi mai ricevuti.

		Nel ciclo infinito viene traslato il buffer di arrivo e viene salvato il nuovo carattere nell'ultima
		posizione.
		Il buffer è quindi passato alla funzione readCommand insieme alla struttura; a seconda dell'intero
		ritornato da quest'ultima funzione viene controllato il comando valido ed invocata la funzione 
		corrispondente, oppure il ciclo riprende da capo.

		Nel caso in cui la funzione readCommand dovesse ritornare una lettura corretta ma non fosse stato
		salvato un comando valido nella struttura, la funzione main restituisce un 1 per segnalare un errore.
	\subsection{read.c and read.h}
		Nel file header read.h sono presenti i comandi $\#$define per aumentare la leggibilità e la manutenibilità
		del codice, insieme con la dichiarazione del tipo "basicCmd", ossia un tipo-struttura  utile per la variabile
		in cui saranno salvati i vari parametri del comando, e la dichiarazione delle funzioni relative alla
		lettura del prossimo carattere dal periferico di input, al riconoscimento del comando e al salvataggio
		dei relativi parametri e alla conversione di un carattere ad intero. 
		\paragraph{readChar(void)}
			La funzione ritorna il carattere che viene letto dal periferico di input, in questo caso lo standard input.
			Viene effettuata una chiamata a funzione "getc()" con arcomento "stdin" (il tutto corrisponde a "getchar()").
		\paragraph{char2int(char charIn)}
			La funzione riceve un carattere in ingresso e restituisce uno short int che corrisponde alla cifra 
			rappresentata in ASCII. Se il carattere non è una cifra da 0 a 9, viene restituito 10, perché è un
			valore che aiuta il parser a capire se i parametri dell'ipotetico comando sono validi oppure no.
		\paragraph{readCommand(char* cmdBuffer, basicCmd* cmdStruc$\_$pt)}
			La funzione riceve l'array buffer in cui è salvato il possibile comando, insieme con la struttura dati in
			cui i parametri del comando vengono salvati.

			La funzione si occupa di controllare nelle posizioni chiave se è presente un identificativo del comando.
			Se è presente, la struttura viene aggiornata grazie all'utilizzo di char2int(char charIn).
			In seguito viene eseguito un controllo sui parametri appena aggiornati: se sono validi, allora viene
			ritornato un 1.

			In caso il comando non sia riconosciuto come completamente valido, viene ritornato un 1.
      \subsection{draw.c, draw.h and shared.h}
      Il file draw.c contiene la definizione e la descrizione dell'interfaccia delle funzioni drawPoint, drawLine e drawEllipse e la direttiva \#include al file locale draw.h.
      IL file draw.h contiene la dichiarazione delle funzioni drawPoint, drawLine e drawEllipse e la definizione di tre macro, tre variabili di nome DRAW\_MODE\_CLEAR, DRAW\_MODE\_SET e DRAW\_MODE\_XOR.
      draw.h ha anche una direttiva \#include al file locale shared.h e una al file di sistema stdio.h.
      IL file shared.h contiene la definizione delle macro rowsFrame, colsFrame e wordPixels e la definizione della variabile globale frameBuffer,
      corrispondende allo schermo virtuale su cui disegnare punti, ellissi e linee a richiesta. frameBuffer viene poi definito nel main.c
		\paragraph{drawPoint}
		\paragraph{drawLine}
                \begin{lstlistings}
                  int drawLine(unsigned int x1,unsigned int y1,unsigned int x2,unsigned int y2,int m)
                \end{lstlistings}
                La funzione drawLine descrive, dati due estremi, un segmento all'interno del frameBuffer.
                \newline
                Riceve, come da prototipo, le coordinate dei due punti (x1,y1) e (x2,y2), ed infine la modalità di disegno.
                Ritorna il valore intero '0' se la scrittura all'interno del frameBuffer o '1' se le coordinate degli estremi sono superiori all dimensioni massime dei ``pixel'' scrivibili.
                L'algoritmo utilizzato per la scrittura del codice è stato ricavato da un documento reperibile al seguente indirizzo web a pagina 6:\newline
                \href{http://www.idav.ucdavis.edu/education/GraphicsNotes/Bresenhams-Algorithm.pdf}\newline
                Lo pseudo-codice si basa sull'algoritmo di Bresenham implementato con numeri interi.
                Tuttavia nel pdf viene considerato il punto P_{1}(x_{1},y_{1}) sempre l'estremo di partenza, dando per scontato che le ordinate e coordinate siano di modulo inferiore a quelle del punto P_{2}(x_{2},y_{2}).
                È stata dunque necessaria una rivisitazione per considerare tutti gli altri casi e permettere un corretto funzionamento del codice.
                La funzione richiede, infine, l'utilizzo di 6 \textit{int} e 1 \textit{char}.
		\paragraph{int drawEllipse(int xc, int xy, int dx, int dy, int m)}
		La funzione drawEllipse riceve cinque variabili intere come parametri:
                le cordinate cartesiane del centro dell'ellisse, i diametri orizontale e verticale e il modo attraverso cui scrivere su un array di char.
		
		drawEllipse è definita nel file draw.c, dichiarata nel file draw.h.
                Fa uso di tre macro xmax, ymax e wordPixels, definite le prime due in read.h, l'ultima in shared.h, tre variabili intere di cui la funzione fa ampio uso.
		
		Essa si serve dell'algoritmo di Bresenham per disegnare sull'array di char.
                Il codice dell'algoritmo è stato ricavato correggendo e modificando la porzione di righe di codice trovata alla pagina web
                \href{https://sites.google.com/site/ruslancray/lab/projects/bresenhamscircleellipsedrawingalgorithm/bresenham-s-circle-ellipse-drawing-algorithm}{Bresenham's circle or ellipse drawing algorithm}.
                L'algoritmo trovato viene corretto usando in totale due variabili locali di tipo int in meno, non usando la funzione DrawPixel()
                ed ottimizzando l'algoritmo in modo che non scriva un pixel più di una volta. Quest'ultima correzione è la più importante:
                senza questa, drawEllipse non sarebbe stata in grado di disegnare secondo il draw mode xor.
                Altra importante correzione riguarda il controllo della posizione del pixel da scrivere.
                Se questa cade al di fuori dello schermo virtuale a disposizione, i bit dell'array di char, il pixel non viene disegnato.
		
		La funzione ritorna uno zero.

\section{Esercizio 2}
	\subsection{Dimensione del codice}
		\paragraph{x86$\_$64}
		\paragraph{ARM}
	\subsection{Script}

\end{document}
